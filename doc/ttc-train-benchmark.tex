\documentclass[submission]{eptcs}
\providecommand{\event}{TTC 2015}

\usepackage[T1]{fontenc}
\usepackage{varioref}
\usepackage{hyperref}

\usepackage{url}
\usepackage{paralist}
\usepackage{graphicx}
\usepackage[cache]{minted}
\newminted{clojure}{fontsize=\fontsize{8}{8},linenos,numbersep=3pt,numberblanklines=false}
\newmintinline{clojure}{fontsize=\small}
\newcommand{\code}{\clojureinline}
\VerbatimFootnotes

\title{Solving the TTC Train Benchmark Case with FunnyQT}
\author{Tassilo Horn
  \institute{Institute for Software Technology, University Koblenz-Landau, Germany}
  \email{horn@uni-koblenz.de}}

\def\titlerunning{Solving the TTC Train Benchmark Case with FunnyQT}
\def\authorrunning{T. Horn}

\begin{document}
\maketitle

\begin{abstract}
  This paper describes the FunnyQT solution to the TTC 2015 Train Benchmark
  transformation case.  The solution solves all core and all extension tasks,
  and it won the \emph{overall quality award}.
\end{abstract}


\section{Introduction}
\label{sec:introduction}

This paper describes the FunnyQT\footnote{\url{http://funnyqt.org}}
~\cite{Horn2013MQWFQ,funnyqt-icgt15} solution of the TTC 2015 Train Benchmark
Case~\cite{train-benchmark-case-desc}.  All core and extension tasks have been
solved.  The solution project is available on
Github\footnote{\url{https://github.com/tsdh/ttc15-train-benchmark-funnyqt}},
and it is set up for easy reproduction on a SHARE image\footnote{The SHARE
  image name is \verb|ArchLinux64_TTC15-FunnyQT_2|}.  This solution won the
\emph{overall quality award} for this case.

FunnyQT is a model querying and transformation library for the functional Lisp
dialect Clojure\footnote{\url{http://clojure.org}}.  Queries and
transformations are Clojure programs using the features provided by the FunnyQT
API.

Clojure provides strong metaprogramming capabilities that are used by FunnyQT
in order to define several \emph{embedded domain-specific languages} (DSL) for
different querying and transformation tasks.

FunnyQT is designed with extensibility in mind.  By default, it supports EMF
models and JGraLab TGraph models.  Support for other modeling frameworks can be
added without having to touch FunnyQT's internals.

The FunnyQT API is structured into several namespaces, each namespace providing
constructs supporting concrete querying and transformation use-cases, e.g.,
model management, functional querying, polymorphic functions, relational
querying, pattern matching, in-place transformations, out-place
transformations, bidirectional transformations, and some more.  For solving the
train benchmark case, especially its in-place transformation DSL has been used.


\section{Solution Description}
\label{sec:solution-description}

In this section, the individual tasks are discussed one by one.  They are all
implemented as in-place transformation rules supported by FunnyQT's
\emph{funnyqt.in-place} transformation DSL.  The rules' repair actions simply
call the CRUD functions of the EMF-specific \emph{funnyqt.emf} namespace.

\paragraph{Task 1: PosLength.}

The transformation rule realizing the \emph{PosLength} task is given below.

\begin{clojurecode}
(defrule pos-length {:forall true :recheck true} [g]
  [segment<Segment>
   :when (<= (eget-raw segment :length) 0)]
  (eset! segment :length (inc (- (eget-raw segment :length)))))
\end{clojurecode}

The \code|defrule| macro defines a new in-place transformation rule with the
given name (\code|pos-length|), an optional map of options
(\code|{:forall true, ...}|) a vector of formal parameters (\code|[g]|), a
pattern (\code|[segment<Segment>...]|), and one or many actions to be applied
to the pattern's matches (\code|(eset! ...)|).  The first formal parameter must
denote the model the rule is applied to, so here the argument \code|g| denotes
the train model when the rule is applied using
\code|(pos-length my-train-model)|.

The pattern matches a node called \code|segment| of metamodel class
\code|Segment|.  Additionally, the segment's length must be less or equal to
zero as defined by the \code|:when| constraint.  The action says that the
segment's \code|length| attribute should be set to the incremented negation of
the current length.

The normal semantics of applying a rule is to find one single match of the
rule's pattern and then execute the rule's actions on the matched elements.
The \code|:forall| option changes this behavior to finding all matches first,
and then applying the actions to each match one after the other.  FunnyQT
automatically parallelizes the pattern matching process of such forall-rules
under certain circumstances like the JVM having more than one CPU available and
the pattern declaring at least two elements to be matched.

The \code|:recheck| option causes the rule to recheck if a pre-calculated match
is still conforming the pattern just before executing the rule's actions on it.
This can be needed for forall-rules whose actions possibly invalidate matches
of the same rule's pattern, e.g., when the application of the action to a match
\(m_i\)
cause another match \(m_j\)
to be no valid match any longer\footnote{This cannot happen for the
  \code|pos-length| rule, however the case description demands matches to be
  revalidated before applying the repair actions.}.


\paragraph{Task 2: SwitchSensor.}

The transformation rule realizing the \emph{SwitchSensor} task is given below.

\begin{clojurecode*}{firstnumber=last}
(defrule switch-sensor {:forall true :recheck true} [g]
  [sw<Switch> -!<:sensor>-> <>]
  (eset! sw :sensor (ecreate! nil 'Sensor)))
\end{clojurecode*}

It matches a switch \code|sw| which is not contained by some sensor.  The
exclamation mark of the edge symbol \code|-!<:sensor>->| specifies that no such
reference must exist, i.e., it specifies a negative application condition.  The
action fixes this problem by creating a new \code|Sensor| and assigning that to
the switch \code|sw|.


\paragraph{Task 3: SwitchSet.}

The \code|switch-set| rule realizes the \emph{SwitchSet} task.  Its definition
is given below.

\begin{clojurecode*}{firstnumber=last}
(def Signal-GO (eenum-literal 'Signal.GO))

(defrule switch-set {:forall true :recheck true} [g]
  [route<Route> -<:entry>-> semaphore
   :when (= (eget-raw semaphore :signal) Signal-GO)
   route -<:follows>-> swp -<:switch>-> sw
   :let [swp-pos (eget-raw swp :position)]
   :when (not= (eget-raw sw :currentPosition) swp-pos)]
  (eset! sw :currentPosition swp-pos))
\end{clojurecode*}

It matches a \code|route| with its entry \code|semaphore| where the semaphore's
signal is \code|Signal.GO|.  The route follows some switch position \code|swp|
whose switch \code|sw|'s current position is different from that of the switch
position.  The fix is to set the switch's current position to the position of
the switch position \code|swp|.

Note that there are no metamodel types specified for the elements
\code|semaphore|, \code|swp|, and \code|sw| because those are already defined
implicitly by the references leading to them, e.g., all elements referenced by
a route's \code|follows| reference can only be instances of
\code|SwitchPosition| according to the metamodel.  FunnyQT doesn't require the
transformation writer to encode tautologies in her patterns\footnote{In fact,
  if there are types specified, those will be checked.  So omitting them when
  they are not needed also results in slightly faster patterns.}.


\paragraph{Extension Task 1: RouteSensor.}

The extension task \emph{RouteSensor} is realized by the \code|route-sensor|
rule given below.

\begin{clojurecode*}{firstnumber=last}
(defrule route-sensor {:forall true :recheck true} [g]
  [route<Route> -<:follows>-> swp -<:switch>-> sw
   -<:sensor>-> sensor --!<> route]
  (eadd! route :definedBy sensor))
\end{clojurecode*}

It matches a \code|route| that follows some switch position \code|swp| whose
switch \code|sw|'s \code|sensor| is not contained by the \code|route|.  The
repair action is to assign the \code|sensor| to the \code|route|.


\paragraph{Extension Task 2: SemaphoreNeighbor.}

The second and last extension task \emph{SemaphoreNeighbor} is realized by the
\code|semaphore-neighbor| rule defined as shown below.

\begin{clojurecode*}{firstnumber=last}
(defrule semaphore-neighbor {:forall true :recheck true} [g]
  [route1<Route> -<:exit>-> semaphore
   route1 -<:definedBy>-> sensor1 -<:elements>-> te1
   -<:connectsTo>-> te2 -<:sensor>-> sensor2
   --<> route2<Route> -!<:entry>-> semaphore
   :when (not= route1 route2)]
  (eset! route2 :entry semaphore))
\end{clojurecode*}

It matches a route \code|route1| which has an exit \code|semaphore|.
Additionally, \code|route1| is defined by a sensor \code|sensor1| which
contains some track element \code|te1| that connects to some track element
\code|te2| whose sensor is \code|sensor2|.  This \code|sensor2| is contained by
some other route \code|route2| which does not have \code|semaphore| as entry
semaphore.  The fix is to set \code|route2|'s entry reference to
\code|semaphore|.


\subsection{Deferred Rule Actions}
\label{sec:deferred-actions}

As mentioned above, the normal semantics of a forall-rule is to compute all
matches of the rule's pattern first (possibly in parallel), and then apply the
rule's actions on every match one after the other.  However, the case
description strictly separates the computation of matches from the repair
transformations.

FunnyQT also provides stand-alone patterns.  Using them, one could have defined
patterns for finding occurrences of the five problematic situations in a train
model, and separate functions for the repair actions where the latter receive
one match of the corresponding pattern and fix that.

But for in-place transformation rules, FunnyQT also provides \emph{rule
  application modifiers}.  Concretely, any in-place transformation rule
\code|r| can be called as \code|(as-pattern (r model))| in which case it
behaves as a pattern.  That is, where a normal rule would usually find one
match and apply its actions on it and a forall-rule would usually find all
matches and apply its actions to each of them, when called with
\code|as-pattern|, a rule simply returns the sequence of its matches.  With a
normal rule, this sequence is a lazy sequence, i.e., the matches are not
computed until they are consumed.  With a forall-rule, the sequence is fully
realized, i.e., all matches are already pre-calculated (possibly in parallel).

The second FunnyQT rule application modifier is \code|as-test|, and this is
what is highly suitable for this transformation case.  When a rule \code|r| is
applied using \code|(as-test (r model))|, it behaves almost as without modifier
but instead of applying the rule's actions immediately, it returns a closure of
arity zero (a so-called \emph{thunk}) which captures the rule's match and the
rule's actions.  Invoking the thunk causes the actions to be applied on the
match.  Thus, the caller of the rule gets the information if the rule was
applicable at all, and if it was applicable, she can decide if she wants to
apply it or not.  And when she applies it, the pattern matching part is already
finished and only the actions are applied on the pre-calculated match the thunk
closes over.

In case of a forall-rule \code|r|, \code|(as-test (r model))| doesn't return a
single thunk but a vector of thunks, one thunk per match of the rule's pattern.
This is exactly what is needed for solving this transformation case.  Using
this feature, a final function is defined that receives a rule \code|r| and a
train model \code|g| and executes the rule as a test.

\begin{clojurecode*}{firstnumber=27}
(defn call-rule-as-test [r g]
  (as-test (r g)))
\end{clojurecode*}

This function is then called with the transformation rules from the Java
trainbenchmark framework.  The given rule gets applied and returns a sequence
of thunks which will apply the actions to the match they are wrapping.  Thus,
the only thing the framework has to do is to apply the thunks corresponding to
the matches which are going to be repaired in the current repair phase.

These 28 lines of Clojure code form the complete functional part of the FunnyQT
solution that solves all core and extension tasks.  There is also a plain-Java
glue project which implements the interfaces required by the benchmark
framework and simply delegates to the Clojure/FunnyQT part of the solution.
This glue project is briefly discussed in the following section.


\subsection{Gluing the Solution with the Framework}
\label{sec:gluing}

Typically, open-source Clojure libraries and programs are distributed as JAR
files that contain the source files rather than byte-compiled class files.
This solution does the same, and that JAR is deployed to a local Maven
repository from which the Maven build infrastructure of the benchmark framework
can pick it up.

Then, in the FunnyQT glue project the rules and functions from above are
referred to like shown in the next listing.

\begin{minted}[fontsize=\fontsize{8}{8},linenos,numbersep=3pt,numberblanklines=false]{java}
private final static String SOLUTION_NS = "ttc15-train-benchmark-funnyqt.core";
Clojure.var("clojure.core", "require").invoke(Clojure.read(SOLUTION_NS));
final static IFn POS_LENGTH = Clojure.var(SOLUTION_NS, "pos-length");
...
final static IFn CALL_RULE_AS_TEST = Clojure.var(SOLUTION_NS, "call-rule-as-test");
\end{minted}

In line 2, the solution namespace \code|ttc15-train-benchmark-funnyqt.core| is
required\footnote{\code|require| is kind of Clojure's equivalent to Java's
  \code|import| statement.}.  The \code|Clojure| class provides a minimal API
for loading Clojure code from Java.  When requiring a namespace as above, it
will be parsed and compiled to JVM byte-code just in time\footnote{If the
  Clojure code was distributed in a pre-compiled form, the resulting classes
  would simply be loaded.}.

Thereafter, the solution's in-place transformation rules and the
\code|call-rule-as-test| function are referred to.  \code|IFn| is a Clojure
interface whose instances are Clojure functions that can be called using the
\code|invoke()| method as can be seen in the definition of the glue project's
\code|BenchmarkCase.check()| method shown below.

\begin{minted}[fontsize=\fontsize{8}{8},linenos,numbersep=3pt,numberblanklines=false]{java}
@Override
protected final Collection<Object> check() throws IOException {
    matches = (Collection<Object>) FunnyQTBenchmarkLogic.CALL_RULE_AS_TEST
                                   .invoke(rule, this.resource);
    // If the rule has no matches it returns nil/null but the framework
    // wants a Collection.
    if (matches == null) {
        matches = new LinkedList<Object>();
    }
    return matches;
}
\end{minted}

In that code, \code|rule| is one of the rule \code|IFn|s \code|POS_LENGTH|,
\code|SWITCH_SET|, et cetera, and they are called via the
\code|call-rule-as-test| function to make them return one thunk per match
instead of performing the rules' repair actions immediately.

The implementation of the \code|BenchmarkCase.modify()| method is even simpler.

\begin{minted}[fontsize=\fontsize{8}{8},linenos,numbersep=3pt,numberblanklines=false]{java}
@Override
protected final void modify(Collection<Object> matches) {
    for (Object m : matches) {
        ((IFn) m).invoke();
    }
}
\end{minted}

Since the rules are called as tests and thus return thunks that apply the
rule's actions, those simply need to be invoked.


\section{Evaluation \& Conclusion}
\label{sec:evaluation}

The FunnyQT solution implements all core and all extension tasks exactly as
demanded by the case description, thus it is \emph{complete}.  When run in the
benchmark framework, all assertion it checks are satisfied, thus the solution
is also \emph{correct}.

The FunnyQT solution consists of 28 NCLOC of FunnyQT/Clojure code for the five
rules with their patterns and repair actions, and the function
\code|call-rule-as-test|.  Therefore, it is very \emph{concise}.

\emph{Readability} is a very subjective matter, and not everyone is fond of
Lisp syntax.  However, there are some strong points with respect to
readability.
\begin{inparaenum}[(1)]
\item The queries (patterns) and repair actions are bundled in one in-place
  transformation rule each keeping the definition of cause and action
  localized.
\item FunnyQT's pattern matching DSL used to specify the rules' patterns is
  both concise and readable.  It should be easy to understand for graph
  transformation experts especially if they have used other textual graph
  transformation languages such as \emph{GrGen.NET} before.  It should also be
  easy to understand for any Clojure programmer because it strictly conforms to
  the style guidelines and best practices there.
\end{inparaenum}

FunnyQT implements pattern matching as a local search.  Thus, each
\emph{recheck} phase take approximately as much time as the initial
\emph{check} phase.  In contrast, with an incremental approach like
\emph{EMF-IncQuery}, the rechecking the patterns is not needed because all
matches of all patterns are cached and updated when the model changes.  This
makes FunnyQT not especially suited for incremental model validation scenarios.
However, given the fact that the evaluation of forall-patterns is automatically
parallelized on multi-core machines, the \emph{performance} is still quite
good.  The benefit of FunnyQT's search-based approach is that it has far less
memory requirements than an incremental approach.


\bibliographystyle{eptcs}
\bibliography{ttc-train-benchmark}

\end{document}

%%% Local Variables:
%%% mode: latex
%%% TeX-master: t
%%% TeX-command-extra-options: "-shell-escape"
%%% LaTeX-verbatim-macros-with-delims-local: ("code")
%%% End:

%  LocalWords:  parallelizes
