\documentclass[submission]{eptcs}
\providecommand{\event}{TTC 2015}

\usepackage[T1]{fontenc}
\usepackage{url}
\usepackage{hyperref}
\usepackage{paralist}
\usepackage[cache]{minted}
\newminted{clojure}{fontsize=\fontsize{8}{8},linenos,numbersep=3pt,numberblanklines=false}
\newmintinline{clojure}{fontsize=\footnotesize}
\newcommand{\code}{\clojureinline}

\title{Solving the TTC Train Benchmark Case with FunnyQT}
\author{Tassilo Horn
  \institute{Institute for Software Technology, University Koblenz-Landau, Germany}
  \email{horn@uni-koblenz.de}}

\def\titlerunning{Solving the TTC Train Benchmark Case with FunnyQT}
\def\authorrunning{T. Horn}

\begin{document}
\maketitle

\begin{abstract}
  FunnyQT is a model querying and model transformation library for the
  functional Lisp-dialect Clojure providing a rich and efficient querying and
  transformation API.  This paper describes the FunnyQT solution to the TTC
  2015 Train Benchmark transformation case which solves all core tasks and all
  extension tasks.
\end{abstract}


\section{Introduction}
\label{sec:introduction}

This paper describes the FunnyQT\footnote{\url{http://funnyqt.org}}
~\cite{Horn2013MQWFQ} solution of the TTC 2015 Train Benchmark
Case~\cite{train-benchmark-case-desc}.  All core and extension tasks have been
solved.  The solution project is available on
Github\footnote{\url{https://github.com/tsdh/ttc15-train-benchmark-funnyqt}},
and it is set up for easy reproduction on a SHARE image\footnote{\url{FIXME}}.

FunnyQT is a model querying and transformation library for the functional Lisp
dialect Clojure\footnote{\url{http://clojure.org}}.  Queries and
transformations are plain Clojure programs using the features provided by the
FunnyQT API.  This API is structured into several task-specific
sub-APIs/namespaces, e.g., there is a namespace \emph{funnyqt.in-place}
containing constructs for writing in-place transformations, a namespace
\emph{funnyqt.model2model} containing constructs for model-to-model
transformations, a namespace \emph{funnyqt.bidi} containing constructs for
bidirectional transformations, and so forth.

As a Lisp, Clojure provides strong metaprogramming capabilities that are
exploited by FunnyQT in order to define several \emph{embedded domain-specific
  languages} (DSL, \cite{book:Fowler2010DSL}) for different tasks.  For
example, the pattern matching/in-place transformation rule constructs used in
this solution is provided in terms of a task-oriented DSL.


\section{Solution Description}
\label{sec:solution-description}

\paragraph{Task 1: PosLength.}

\begin{clojurecode}
(defrule ^:forall ^:recheck pos-length [g]
  [segment<Segment>
   :when (<= (eget-raw segment :length) 0)]
  (eset! segment :length (inc (- (eget-raw segment :length)))))
\end{clojurecode}


\paragraph{Task 2: SwitchSensor.}

\begin{clojurecode*}{firstnumber=5}
(defrule ^:forall ^:recheck switch-sensor [g]
  [sw<Switch> -!<:sensor>-> <>]
  (eset! sw :sensor (ecreate! nil 'Sensor)))
\end{clojurecode*}


\paragraph{Task 3: SwitchSet.}

\begin{clojurecode*}{firstnumber=8}
(def Signal-GO (eenum-literal 'Signal.GO))

(defrule ^:forall ^:recheck switch-set [g]
  [route<Route> -<:entry>-> semaphore
   :when (= (eget-raw semaphore :signal) Signal-GO)
   route -<:follows>-> swp -<:switch>-> sw
   :let [cur-pos (eget-raw swp :position)]
   :when (not= (eget-raw sw :currentPosition) cur-pos)]
  (eset! sw :currentPosition cur-pos))
\end{clojurecode*}


\paragraph{Extension Task 1: RouteSensor.}

\begin{clojurecode*}{firstnumber=16}
(defrule ^:forall ^:recheck route-sensor [g]
  [route<Route> -<:follows>-> swp -<:switch>-> sw
   -<:sensor>-> sensor --!<> route]
  (eunset! sw :sensor))
\end{clojurecode*}


\paragraph{Extension Task 2: SemaphoreNeighbor.}

\begin{clojurecode*}{firstnumber=20}
(defrule ^:forall ^:recheck semaphore-neighbor [g]
  [route1<Route> -<:exit>-> semaphore
   route1 -<:definedBy>-> sensor1 -<:elements>-> te1
   -<:connectsTo>-> te2 -<:sensor>-> sensor2
   --<> route2<Route> -!<:entry>-> semaphore
   :when (not= route1 route2)]
  (eunset! route1 :exit))
\end{clojurecode*}


\paragraph{Calling rules with deferred execution of actions.}

\begin{clojurecode*}{firstnumber=27}
(defn call-rule-as-test [r g]
  (as-test (r g)))
\end{clojurecode*}


\paragraph{Comparing matches.}

\begin{clojurecode*}{firstnumber=29}
(defn make-match-comparator [& kws]
  (fn [t1 t2]
    (loop [kws kws]
      (if (seq kws)
        (let [m1 ((first kws) (:match (meta t1)))
              m2 ((first kws) (:match (meta t2)))]
          (let [r (compare (eget (first m1) :id)
                           (eget (first m2) :id))]
            (if (zero? r)
              (recur (rest kws))
              r)))
        (errorf "These two matches compare to zero: %s %s"
                (:match (meta t1))
                (:match (meta t2)))))))
\end{clojurecode*}

\section{Evaluation and Conclusion}
\label{sec:evaluation}

\bibliographystyle{eptcs}
\bibliography{ttc-train-benchmark}
\end{document}

%%% Local Variables:
%%% mode: latex
%%% TeX-master: t
%%% TeX-command-extra-options: "-shell-escape"
%%% End:
